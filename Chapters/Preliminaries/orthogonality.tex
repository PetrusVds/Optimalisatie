\begin{theo}[Orthogonality]{Orthogonality}
    A set of nonzero vectors $a_1, a_2, \ldots, a_n$ in $\R^n$ is said to be orthogonal if the dot product of any two distinct vectors is zero, i\@.e\@.
    \begin{equation*}
        \forall i,j \in [1,n]: \ i \neq j \ \Rightarrow \ a_i \cdot a_j = 0
    \end{equation*}
    Consequently, two subspaces $S_1$ and $S_2$ of $\R^n$ are orthogonal if every vector in $S_1$ is orthogonal to every vector in $S_2$. In that case, $S_2$ is called the orthogonal complement of $S_1$ and denoted by $S_1^{\perp}$. The following now holds true for a subspace $S$ of $\R^n$:
    \begin{equation*}
        \R^n = S \oplus S^{\perp}
    \end{equation*}
    \vspace*{-0.5cm}
\end{theo}

\begin{theo}[Orthonormality]{Orthonormality}
    A set of nonzero vectors $a_1, a_2, \ldots, a_n$ in $\R^n$ is said to be orthonormal if it is orthogonal and each vector has a unit length, i\@.e\@.
    \begin{equation*}
        \forall i,j \in [1,n]: \ i \neq j \ \Rightarrow \ a_i \cdot a_j = 0 \ \land \ \|a_i\| = 1
    \end{equation*}
    \vspace*{-0.5cm}
\end{theo}