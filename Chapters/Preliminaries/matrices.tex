\begin{ex}[Types of matrices]{TypesMatrices}
    A matrix $A \in \R^{m \times n}$ is said to be
    \begin{itemize}
        \item 
            the zero matrix, denoted by $0$, if all its entries are zero,
        \item 
            a square matrix if $m = n$,
        \item 
            the identity matrix if it is square and all its diagonal entries are one,
        \item 
            a diagonal matrix if all its off-diagonal entries are zero,
        \item 
            an upper triangular matrix if all its entries below the diagonal are zero,
        \item 
            a lower triangular matrix if all its entries above the diagonal are zero,
        \item
            a symmetric matrix if it is square and $A = A^T$, the set of these matrices is denoted by $\mathbb{S}^n$,
        \item 
            an orthogonal matrix if it is square and $AA^T = A^T A = I$,
        \item 
            a non-singular matrix if it is square and there exists another square matrix $B \in \R^{n \times n}$, the inverse of $A$, such that 
            \begin{equation*}
                AB = BA = I
            \end{equation*} 
        \item 
            a dyadic matrix if it is of the form $A = uv^T$ for some vectors $u, v \in \R^n$.
    \end{itemize}
\end{ex}

\begin{theo}[Range]{Range}
    Given a matrix $A \in \R^{m \times n}$, the range of $A$ is the set of $m$-dimensional vectors that can be expressed as $Ax$ for some $n$-dimensional vector $x$, and we denote it by 
    \begin{equation*}
        \mathcal{R}(A) = \{ Ax \ | \ x \in \R^n \}
    \end{equation*}
    In other words, it is the set of vectors that can be expressed as linear combinations of the columns of $A$.
    % \vspace*{-0.5cm}
\end{theo}

\begin{theo}[Kernel]{Kernel}
    Given a matrix $A \in \R^{m \times n}$, the kernel of $A$ is the set of $n$-dimensional vectors that are mapped to the zero vector by $A$, and we denote it by
    \begin{equation*}
        \mathcal{N}(A) = \{ x \in \R^n \ | \ Ax = 0 \}
    \end{equation*}
    In other words, it is the set of vectors that are orthogonal to the columns of $A$.
    % \vspace*{-0.5cm}
\end{theo}

\newpage

\begin{theo}[Rank]{Rank}
    Given a matrix $A \in \R^{m \times n}$, the rank of $A$ is the dimension of its range, i\@.e\@.
    \begin{equation*}
        \text{rank}(A) = \dim(\mathcal{R}(A))
    \end{equation*}
    \textbf{Note:} The sum of the dimensions of the range of $A$ and the null space (kernel) of $A$ is equal to the number of columns $n$:
    \begin{equation*}
        \dim(\mathcal{R}(A)) + \dim(\mathcal{N}(A)) = n
    \end{equation*}
    \vspace{-0.5cm}
\end{theo}

\begin{theo}[Fundamental theorem of Linear Algebra]{FunLinAlg}
    For any matrix $A \in \R^{m \times n}$, it holds that $\mathcal{N}(A) \perp \mathcal{R}(A^T)$ and $\mathcal{N}(A^T) \perp \mathcal{R}(A)$, therefore we have 
    \begin{align*}
        \R^n &= \mathcal{N}(A) \oplus \mathcal{R}(A^T) \\
        \R^m &= \mathcal{N}(A^T) \oplus \mathcal{R}(A)
    \end{align*}
    \vspace*{-0.5cm}
\end{theo}

\begin{theo}[Singular Value Decomposition]{SVD}
    Every matrix $A \in \R^{m \times n}$ of rank $r$ can be written as
    \begin{equation*}
        A = U \Sigma V^T = \left[
            \begin{array}{cc}
            U_1 & U_2
            \end{array}
            \right]
            \left[
            \begin{array}{cc}
            \Sigma_1 & 0 \\
            0 & 0
            \end{array}
            \right]
            \left[
            \begin{array}{c}
            V_1^\top \\
            V_2^\top
            \end{array}
            \right]
    \end{equation*}
    where $U \in \R^{m \times m}$ and $V \in \R^{n \times n}$ are orthogonal matrices, $U_1 \in \R^{m \times r}$ , $V_1 \in \R^{n \times r}$ and 
    \begin{equation*}
        \Sigma_1 = \text{diag}(\sigma_1, \sigma_2, \ldots, \sigma_r) \in \R^{r \times r}
    \end{equation*}
    is a diagonal matrix, where $\sigma_1 \geq \sigma_2 \geq \ldots \geq \sigma_r > 0$ are the singular values of $A$. The columns of $U$ and $V$ are called the left and right singular vectors of $A$, respectively.
\end{theo}

\begin{pro}[Singular Value Decomposition]{proSVD}
    The singular value decomposition of a matrix $A \in \R^{m \times n}$ gives orthogonal bases for the four fundamental subspaces related to A:
    \begin{align*}
        \mathcal{R}(A) &= \mathcal{R}(U_1), \quad \mathcal{N}(A^T) = \mathcal{R}(U_2) \\
        \mathcal{R}(A^T) &= \mathcal{R}(V_1), \quad \mathcal{N}(A) = \mathcal{R}(V_2) \\
    \end{align*}
    \vspace{-1cm}
\end{pro}

\begin{theo}[Orthogonal-triangular decomposition (QR)]{QR}
    If $A \in \R^{m \times n}$, then there exists an orthogonal matrix $Q \in \R^{m \times m}$ and an upper triangular matrix $R \in \R^{m \times n}$ such that
    \begin{equation*}
        A = QR.
    \end{equation*}
    \vspace{-0.5cm}
\end{theo}

\newpage

\begin{theo}[Eigenvalue decomposition]{EeigenvalueDecomp}
    Any real symmetric matrix $A \in \R^{n \times n}$ can be decomposed as
    \begin{equation*}
        A = Q \Lambda Q^T,
    \end{equation*}
    where $Q \in \R^{n \times n}$ is orthogonal, i\@.e\@. $Q^T Q = I$, and $\Lambda \in \R^{n \times n}$ is diagonal with the eigenvalues of $A$ on the diagonal. The columns of $Q$ form an orthonormal set of eigenvectors.
\end{theo}

\begin{theo}[Symmetric positive semi-definite matrices]{SymmetricPositiveSemi-definiteMatrices}
    We write for a symmetric matrix $B = B^T$, $B \in \R^{n \times n}$ that ``$B \succeq 0$'' if and only if $B$ is positive semi-definite, i\@.e\@., 
    \begin{equation*}
        \forall z \in \R^n: \ z^T B z \geq 0,
    \end{equation*}
    or, equivalently, if all (real) eigenvalues of $B$ are non-negative.The set of all symmetric positive semi-definite matrices is denoted by $\mathbb{S}^n_+$. 
\end{theo}

\begin{pro}[Symmetric positive semi-definite matrices]{proSymmetricPositiveSemi-definiteMatrices}
    Let $Q \in \mathbb{S}^n$ be a symmetric matrix. Then the following statements are equivalent:
    \begin{enumerate}
        \item $Q$ is positive semi-definite, i\@.e\@. $Q \succeq 0$,
        \item all eigenvalues of $Q$ are non-negative, i\@.e\@. $\lambda_i(Q) \geq 0$ for all $i \in [1,n]$,
        \item all principal minors of $Q$, i\@.e\@. the determinant of a submatrix obtained from $Q$ when the same set of rows and columns are stricken out, are non-negative,
        \item $Q$ can be written as $Q = AA^T$ for some matrix $A \in \R^{n \times r}$ and $r$ is the rank of $Q$.
    \end{enumerate}
    \vspace*{-0.3cm}
\end{pro}

\begin{theo}[Symmetric positive definite matrices]{SymmetricPositiveDefiniteMatrices}
    A symmetric is positive definite if $B \succ 0$, i\@.e\@.,
    \begin{equation*}
        \forall z \in \R^n \backslash \{0\}: \ z^T B z > 0,
    \end{equation*}
    and the set of symmetric positive definite matrices is denoted by $\mathbb{S}^n_{++}$.
\end{theo}

\begin{theo}[Cholesky factorization]{Cholesky}
    If $A \in \R^{n \times n}$ is symmetric positive definite, then there exists a unique lower triangular matrix $L \in \R^{n \times n}$ with positive diagonal entries such that
    \begin{equation*}
        A = LL^T.
    \end{equation*}
    \vspace*{-0.5cm}
\end{theo}

\begin{theo}[Matrix norms]{MatrixNorms}
    For a matrix $A \in \R^{m \times n}$ and two vector norms $\|\cdot\|_p$ and $\|\cdot\|_q$, the induced matrix norm is defined as 
    \begin{equation*}
        \|A\|_{p,q} = \max_{x} \{ \|Ax\|_q \|x\|_p \leq 1 \}.
    \end{equation*}
    When $p = q$, we simply write $\|A\|_p$. 
\end{theo}

\begin{ex}[Spectral norm]{SpectralNorm}
    The $\ell_2$-induced norm or spectral norm of a matrix $A \in \R^{m \times n}$ is defined as
    \begin{equation*}
        \|A\|_2 = \sqrt{\lambda_{\max}(A^T A)} = \sigma_{\max}(A),
    \end{equation*}
    where $\lambda_{\max}(A^T A)$ denotes the largest eigenvalue of $A^T A$ and $\sigma_{\max}(A)$ is the largest singular value of $A$.
\end{ex}

\begin{ex}[Frobenius norm]{Frobenius norm}
    The Frobenius norm of a matrix $A \in \R^{m \times n}$ is defined as
    \begin{equation*}
        \|A\|_F = \sqrt{\sum_{i=1}^m \sum_{j=1}^n a_{ij}^2} = \sqrt{\text{tr}(A^T A)}.
    \end{equation*}
    \vspace{-0.5cm}
\end{ex}

% This is probably unnecessary
% \begin{ex}[Nuclear norm]{Nuclear norm}
%     The nuclear norm of a matrix $A \in \R^{m \times n}$ is defined as
%     \begin{equation*}
%         \|A\|_* = \sum_{i=1}^{\min\{m,n\}} \sigma_i(A),
%     \end{equation*}
%     where $\sigma_i(A)$ denotes the $i$-th singular value of $A$.
% \end{ex}