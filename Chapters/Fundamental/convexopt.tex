\begin{theo}[Convexity for $C^1$ functions]{ConvexityC1}
    \begin{minipage}{0.60\textwidth}
        Assume that $f: \ \Omega \rightarrow \R$ is continuously differentiable and $\Omega$ is convex. Then holds that $f$ is convex if and only if 
        \begin{equation*}
            \forall x,y \in \Omega: \ f(y) \geq f(x) + \nabla f(x)^T(y-x)
        \end{equation*}
        i\@.e\@. tangents lie below the graph.
    \end{minipage}
    \begin{minipage}{0.3\textwidth}
        \begin{center}
            \includegraphics[scale = 0.45]{Images/Fundamental/C1Convexity.png}
        \end{center}
    \end{minipage}
\end{theo}

\begin{prf}[Convexity for $C^1$ functions]{prfConvexityC1}
    ``$\Rightarrow$'': Due to convexity of $f$ holds for given $x,y \in \Omega$  and for any $\lambda \in [0,1]$ that
    \begin{equation*}
        f(x + \lambda(y-x)) - f(x) \leq \lambda(f(y) - f(x)).
    \end{equation*}
    and therefore that 
    \begin{equation*}
        \nabla f(x)^T(y-x) 
            = \lim_{\lambda \rightarrow 0}  \frac{f(x + \lambda(y-x)) - f(x)}{\lambda}
            \leq \frac{f(y) - f(x)}{y-x}.
    \end{equation*}

    ``$\Leftarrow$'': To prove that $z = x + \lambda(y-x) = (1-\lambda)x + \lambda y$ holds that $f(z) \leq (1-\lambda)f(x) + \lambda f(y)$, we can use the equation from Theorem~\ref{ConvexityC1} twice to get
    \begin{equation*}
        f(x) \geq f(z) + \nabla f(z)^T(x-z) \ \ \text{and} \ \ f(y) \geq f(z) + \nabla f(z)^T(y-z),
    \end{equation*}
    which yield, when weighted with $(1-\lambda)$ and $\lambda$ respectively, that
    \begin{equation*}
        (1-\lambda)f(x) + \lambda f(y) \geq f(z) + \nabla f(z)^T \underset{= 0}{\underbrace{\left[(1-\lambda)(x-z) + \lambda(y-z)\right]}}
    \end{equation*}
    \vspace{-0.75cm}
\end{prf}

\begin{theo}[Generalized Inequality for Symmetric Matrices]{GeneralizedInequalityForSymmetricMatrices}
    We write for a symmetric matrix $B = B^T$, $B \in \R^{n \times n}$ that ``$B \succeq 0$'' if and only if $B$ is positive semi-definite, i\@.e\@., 
    \begin{equation*}
        \forall z \in \R^n: \ z^T B z \geq 0,
    \end{equation*}
    or, equivalently, if all (real) eigenvalues of $B$ are non-negative.
\end{theo}

\begin{theo}[$O(\cdot)$]{BigODot}
    For a function $f: \R^n \rightarrow \R^m$, we write 
    \begin{equation*}
        f(x) = O(g(x))
    \end{equation*} 
    if and only if there exists a constant $C > 0$ and a neighborhood $\mathcal{N}$ of $0$ such that 
    \begin{equation*}
        \forall x \in \mathcal{N}: \ \| f(x) \| \leq C g(x),
    \end{equation*} 
    i\@.e\@. ``$f$ shrinks as fast as $g$''.
\end{theo}

\begin{theo}[$o(\cdot)$]{SmallODot}
    For a function $f: \R^n \rightarrow \R^m$, we write 
    \begin{equation*}
        f(x) = o(g(x))
    \end{equation*} 
    if and only if there exists a neighborhood $\mathcal{N}$ of $0$ and a function $c: \mathcal{N} \rightarrow \R$ with $\lim_{x \rightarrow 0} c(x) = 0$ such that 
    \begin{equation*}
        \forall x \in \mathcal{N}: \ \| f(x) \| \leq c(x) g(x),
    \end{equation*} 
    i\@.e\@. ``$f$ shrinks faster than $g$''.
\end{theo}

\begin{theo}[Convexity for $C^2$ functions]{ConvexityC2}
    Assume that $f: \ \Omega \rightarrow \R$ is twice continuously differentiable and $\Omega$ is convex and open. Then holds that $f$ is convex if and only if
    \begin{equation*}
        \forall x \in \Omega: \ \nabla^2 f(x) \succeq 0,
    \end{equation*}
    i\@.e\@. the Hessian of $f$ is positive semi-definite.
\end{theo}

\begin{prf}[Convexity for $C^2$ functions]{prfConvexityC2}
    To prove that the Theorem~\ref{ConvexityC1} implies the Theorem~\ref{ConvexityC2}, we can use a second order Taylor expansion of $y$ at $x$ in an arbitrary direction $p$:
    \begin{equation*}
        f(x + tp) = f(x) + t \nabla {f(x)}^T p + \frac{t^2}{2} p^T \nabla^2 f(x) p + o(t^2\|p\|).
    \end{equation*}
    From this we obtain that
    \begin{equation*}
        p^T \nabla^2 f(x) p = \lim_{t \rightarrow 0} \frac{2}{t^2} \left(\underset{(\ref*{ConvexityC1}): \ \geq 0}{\underbrace{f(x + tp) - f(x) - t \nabla {f(x)}^T p}}\right) \geq 0.
    \end{equation*}
    Conversely, to prove the other direction, we use the Taylor rest term formula with some arbitrary $\theta \in (0,1)$:
    \begin{equation*}
        f(y) = f(x) + \nabla {f(x)}^T(y-x) + \underset{(\ref*{ConvexityC2}): \ \geq 0}{\underbrace{\frac{1}{2}t^2 {(y-x)}^T \nabla^2 f(x + \theta(y-x)) (y-x)}}. 
    \end{equation*}
    \vspace{-0.7cm}
\end{prf}

\begin{pro}[Convexity perserving operations on convex functions]{ConvexPerservingOptsConvexFunctions}
    The following operations preserve the convexity of a function:
    \begin{enumerate}
        \item 
            Affine input transformation: If $f: \ \Omega \rightarrow \R$ is convex, then also
            \begin{equation*}
                A \in \R^{n \times m}: \ \tilde{f}(x) = f(Ax + b)
            \end{equation*}
            is convex on the domain $\tilde{\Omega} = \{x \ | \ Ax + b \in \Omega\}$.
        \item 
            Concatenation with monotone convex function: If $f: \ \Omega \rightarrow \R$ is convex and $g: \ \R \rightarrow \R$ is convex and monotonely increasing, then the composition $g \circ f$ is convex.
        \item 
            The supremum over a set of convex functions $f_i(x)$, $i \in I$, i\@.e\@.,
            \begin{equation*}
                f(x) = \sup_{i \in I} f_i(x)
            \end{equation*}
            is convex.
    \end{enumerate}
    \vspace*{-0.2cm}
\end{pro}

\begin{prf}[Convexity perserving operations on convex functions]{prfConvexPerservingOptsConvexFunctions}
    \begin{enumerate}
        \item 
            Not seen in this course.
        \item 
            Recall that $g$ is a convex and monotonely increasing function, then: 
            \begin{equation*}
                \nabla^2 (g \circ f) = \underbrace{g''(f(x))}_{\geq 0} \underbrace{\nabla f(x) \nabla {f(x)}^T }_{\succeq 0} + \underbrace{g'(f(x))}_{\geq 0} \underbrace{\nabla^2 f(x)}_{\succeq 0} \succeq 0,
            \end{equation*}
            i\@.e\@. $g \circ f$ is convex, since the Hessian is positive semi-definite.
        \item 
            Epigraph of $f$ is the intersection of the epigraphs of $f_i$, which are convex.
    \end{enumerate}
    \vspace*{-0.2cm}
\end{prf}

\begin{theo}[Concave function]{Concave}
    A function $f: \ \Omega \rightarrow \R$ is concave if and only if $-f$ is convex.
\end{theo}

% \begin{theo}[Convex Maximization Problem]{ConvexMaximizationProblem}
%     A maximization problem
%     \begin{maxi*}|l|
%         {x \in \mathbb{R}^n}{f(x)}
%         {}{}
%         \addConstraint{x \in \Omega}.
%     \end{maxi*}
%     is called a convex maximization problem if $f$ is concave and $\Omega$ is convex. Naturally, the dual problem is then a convex minimization problem of the form 
%     \begin{mini*}|l|
%         {x \in \mathbb{R}^n}{-f(x)}
%         {}{}
%         \addConstraint{x \in \Omega}.
%     \end{mini*}
%     \vspace*{-0.5cm}
% \end{theo}

% \newpage

\begin{theo}[Convexity of Sublevel subsets]{ConvexitySublevelSubset}
    \vspace*{-0.5cm}
    \begin{minipage}{0.58\textwidth}
        The sublevel set 
        \begin{equation*}
            \{ \ x \in \Omega \ | \ f(x) \leq c \ \}
        \end{equation*} 
        of a convex function $f: \Omega \rightarrow \R$ with respect to any constant $c \in \R$ is convex. 
    \end{minipage}
    \begin{minipage}{0.38\textwidth}
        \begin{center}
            \vspace*{0.5cm}
            \includegraphics[scale = 0.475]{Images/Fundamental/ConvexSubset.png}
        \end{center}
    \end{minipage}
    \vspace{-0.3cm}
\end{theo}

\begin{prf}[Convexity of Sublevel subsets]{prfConvexitySublevelSubset}
    If $f(x) \leq c$ and $f(y) \leq c$, then for any $\lambda \in [0,1]$ holds also 
    \begin{equation*}
        f((1-\lambda)x + \lambda y) \leq (1-\lambda)f(x) + \lambda f(y) \leq \underset{= c}{\underbrace{(1-\lambda)c + \lambda c}}.
    \end{equation*}
    \vspace*{-0.7cm}
\end{prf}

\begin{pro}[Convexity perserving operations on convex sets]{ConvexityPreservingOperations}
    The following operations preserve the convexity of a set:
    \begin{enumerate}
        \item The intersection of finitely or infinitely many convex sets is convex.
        \item Affine image: if $\Omega$ is convex, then for $A \in \R^{m \times n}$, $b \in \R^m$ also the set $A\Omega + b = \{ y \in \R^m \ | \ \exists x \in \Omega : y = Ax + b \}$ is convex.
        \item Affine pre-image: if $\Omega$ is convex, then for $A \in \R^{n \times m}$, $b \in \R^n$ also the set $\{ z \in \R^m \ | \ Az + b \in \Omega \}$ is convex.
    \end{enumerate}
\end{pro}

\begin{ex}[Convex Feasible set]{ConvexFeasibleSet}
    If $\forall i \in [1,m]: f_i : \ \R^n \rightarrow \R$ are convex functions, then the set 
    \begin{equation*}
        \Omega = \{x \in \R^n \ | \ f_i(x) \leq 0, \ i \in [1,m]\}
    \end{equation*}
    is a convex set, because it is the intersection of sublevel sets $\Omega_i$ of convex functions $f_i$, i\@.e\@.
    \begin{align*}
        \Omega 
            &= \bigcap_{i=1}^m \Omega_i \\
            &= \bigcap_{i=1}^m \ \{x \in \R^n \ | \ f_i(x) \leq 0 \}.
    \end{align*}
    \vspace{-0.5cm}
\end{ex}

